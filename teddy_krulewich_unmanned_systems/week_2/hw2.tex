\documentclass{article}

\author{Teddy Krulewich}
\title{\vspace{-4em}HW2 ME5501 – Robotics and Unmanned Systems}

\usepackage{graphicx}
\graphicspath{ {images/} }


\usepackage[utf8]{inputenc}
\usepackage{minted}
\usepackage{hyperref}

\begin{document}
\maketitle


\section*{Problem 1}

Create a function that checks if the current node is valid based upon the list of obstacles, grid 
boundaries, and current location. 
Using an obstacle list of (1,1), (4,4), (3,4), (5,0), (5,1), (0,7), (1,7), (2,7), and (3,7); and a bounding box 
of 0 to 10 for both x and y, and step size of 0.5, verify that the location (2,2) is valid. Assume the 
obstacles have a diameter of 0.5 (only occupy the node at which they reside).
Pass the obstacle list, node, and map boundaries/step size, and return a Boolean True/False 
depending on if the node location is valid (reachable).
Submit your Python code


I chose to invalid nodes in c-space when they are added to the grid using a bounding box system.
This is more efficient as we dont have to check the entire space for invalid nodes and adding obstacles
doesnt path finding time. The code is below:

\end{document}